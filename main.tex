\documentclass[a4paper, 11pt]{article}
    \usepackage{comment} % enables the use of multi-line comments (\ifx \fi) 
    \usepackage{lipsum} %This package just generates Lorem Ipsum filler text. 
    \usepackage{fullpage} % changes the margin
    \usepackage{hyperref} % To insert URL

    
    \begin{document}
    %Header-Make sure you update this information!!!!
    \noindent
    \large\textbf{NDN-RIOT Package Report} \\ 
    \normalsize Tianyuan Yu \\ royu29@ucla.edu
    
    \section{Notes}
    Package is not well-written because I'm not a coding expert, so any suggestions about coding or design you can provide are highly helpful! I'm keep tidying up code and revising neccessary documentation. Apology again if my not so good coding style confuses you.
    
    \section{Package History/Overview}
    NDN-RIOT package is based on Wentao Shang's same named work in 2016, you can find the original paper work via NDN's publication list, or resort to link \url{https://named-data.net/wp-content/uploads/2015/01/design_implementation_ndn_protocol.pdf}.\\ Wentao's work provide NDN protocol stack on RIOT OS, but limited on basic Interest/Data exchanges. Original package creates a \texttt{ndn} thread aside from the user's main thread, serving networking things. Whereas, IoT scenario need built-in app-layer protocols (e.g., bootstrapping, service discovery) to facilitate development. The new package additionally create a \texttt{ndn-helper} thread to interact with core \texttt{ndn} thread, registering faces, fib entries, etc. User can call \texttt{ndn-helper} function to retrieve issued certificate, neighbour identities and available services, and allocated access keys.
    
    \section{Environment Setting}
    \subsection{Source}
    Use RIOT OS from \url{https://github.com/named-data-iot/RIOT} (not the official RIOT OS)\\
    Use NDN-RIOT package from \url{https://github.com/Zhiyi-Zhang/ndn-riot}
    \subsection{Package Configuration}
    RIOT cloned from address above have already equipped with NDN-RIOT package (new old version of Wentao's work in 2016). To replace the configuration, go to folder /RIOT/pkg/ndn-riot, redirect the makefile here to a local source folder, or to remote github link (be sure of using the newest commit version number in makefile) 
    \subsection{Makefile}
    Each new project which uses this package should add

    \section{Test and Examples}
    Basic APIs inherited from Wentao's original library can be found in ndn-riot-examples. But tests for each module is still missing. Protocols design can be found in repo's wiki page. Bootstrapping protocol is little complicated since we optimized it many times for speed issue.      
    Examples can be found in \url{https://github.com/Zhiyi-Zhang/ndn-riot-tests/examples} 

    \section{Core Helper APIs}
    Listed APIs can be found in \texttt{helper-app.h}
    \begin{enumerate}
    \item \texttt{ndn-helper-init()} \\
    Create and Initialize the \texttt{ndn-helper} thread 
    \item \texttt{ndn-helper-bootstrap-start()} \\
    Passing a key pair structure to the function and start bootstrap process with a pre-configured controller. This function will create a thread \texttt{ndn-helper-bootstrap}.To notice, current NDN-RIOT uses ECDSA curve secp160r1. Please make sure the input key pair are the correct type
    \item \texttt{ndn-helper-bootstrap-info()} \\ 
    Retriece the issued certificate from \texttt{ndn-helper}
    \item \texttt{ndn-helper-discovery-init()} \\
    Create and Initialize the thread \texttt{ndn-helper-discovery}, as well as the neighbour table
    \item \texttt{ndn-helper-discovery-register-prefix()} \\
    Register a subprefix to \texttt{ndn-helper-discovery}, this function is called before \texttt{ndn-helper-discovery-init()} and \texttt{ndn-helper-discovery-start()}. All subprefix should be registered before \texttt{ndn-helper-discovery-start()}
    \item \texttt{ndn-helper-discovery-start()} \\
    Start periodically broadcast the registered services and automatically collect the listened neighbour identities and service names to neighbour table. User can check neighbour table via APIs in \texttt{neighbour-table.h}
    \item \texttt{ndn-helper-access-init()} \\
    Create and Initialize the \texttt{ndn-helper-access} thread 
    \item \texttt{ndn-helper-access-terminate()} \\
    Terminate \texttt{ndn-helper-access}
    \item \texttt{ndn-helper-access-producer()} \\
    Apply for access control as a producer, return a pointer of allocated key for content encrytion. This operation need a Access Controller in the network to negotiate symmetric key for applicants.
    \item \texttt{ndn-helper-access-consumer()} \\
    Apply for certain producer identity's access, return a pointer of producer's encrytion key.
    \end{enumerate}
    
    \section{Bootstrapping}
    With ndn-helper, native MacOS/Linux process or boards can perform a bootstrapping client role, but not the server part. Bootstrapping controller need configuration manually. Source code for bootstrapping controller can be seen at ndn-riot-examples. Such consideration is because we plan to re-implement the bootstrapping controller part over Android, where device are powerful enough to generate key pairs with enough security level. The similar situation also exist in access control module.
    
    \section{Service Discovery}
    Neighbour Table is only used in Service Discovery, to automatically collect available identities and services under these prefixes. Table will only be initiated once when you initiate the discovery thread. You can manually add/remove entries of the table if you need.
    
    \section{Access Control}
    Like bootstrapping, ndn-helper can only delegates the identity applying for access control or access keys. Access controller in the network need configuration manually. Source code in the repo ndn-riot-examples.

    
    
    
    \end{document}
    